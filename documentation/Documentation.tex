\documentclass{article}  

%----------------------------------------------------------------------------------------
%	PACKAGES AND OTHER DOCUMENT CONFIGURATIONS
%----------------------------------------------------------------------------------------

\usepackage{fancyhdr}   
\usepackage{lastpage}   
\usepackage{extramarks} 
\usepackage[usenames,dvipsnames]{color} 
\usepackage{graphicx} 
\usepackage{listings} 
\usepackage{courier} 
\usepackage{amssymb}
\usepackage{amsthm}
\usepackage{mathtools}
\usepackage{enumitem}
\usepackage[protrusion=true,expansion=true]{microtype}
\usepackage[svgnames]{xcolor}
\usepackage{fix-cm}
\usepackage{sectsty}
\allsectionsfont{\usefont{OT1}{phv}{b}{n}}
\usepackage{fancyhdr}
\pagestyle{fancy}

% Margins
\topmargin=-0.45in
\evensidemargin=0in
\oddsidemargin=0in
\textwidth=6.5in
\textheight=9.0in
\headsep=0.25in

\linespread{1.1} % Line spacing

\def\dj{d{\hspace{-.32em}\raisebox{.7ex}{-}}}
\def\Dj{D{\hspace{-.75em}\raisebox{.3ex}{-}\hspace{.4em}}}
\def\a{\hspace{.2em},\hspace{-0.17cm},\hspace{.3em}}

\sectionfont{\color{Red}}
\newcommand{\IVemph}[1]{\color{Navy}{\textit{#1}}}

%----------------------------------------------------------------------------------------
%	CODE INCLUSION CONFIGURATION
%----------------------------------------------------------------------------------------

\definecolor{MyDarkGreen}{rgb}{0.0,0.4,0.0} % This is the color used for comments
\lstloadlanguages{Perl} % Load Perl syntax for listings, for a list of other languages supported see: ftp://ftp.tex.ac.uk/tex-archive/macros/latex/contrib/listings/listings.pdf
\lstset{language=Perl,  % Use Perl in this example
        frame=single,   % Single frame around code
        basicstyle=\small\ttfamily,      % Use small true type font
        keywordstyle=[1]\color{Blue}\bf, % Perl functions bold and blue
        keywordstyle=[2]\color{Purple},  % Perl function arguments purple
        keywordstyle=[3]\color{Blue}\underbar, % Custom functions underlined and blue
        identifierstyle=,                      % Nothing special about identifiers
        commentstyle=\usefont{T1}{pcr}{m}{sl}\color{MyDarkGreen}\small, % Comments small dark green courier font
        stringstyle=\color{Purple},                                     % Strings are purple
        showstringspaces=false,                                         % Don't put marks in string spaces
        tabsize=5,                                                      % 5 spaces per tab
        %
        % Put standard Perl functions not included in the default language here
        morekeywords={rand},
        %
        % Put Perl function parameters here
        morekeywords=[2]{on, off, interp},
        %
        % Put user defined functions here
        morekeywords=[3]{test},
       	%
        morecomment=[l][\color{Blue}]{...}, % Line continuation (...) like blue comment
        numbers=left, % Line numbers on left
        firstnumber=1, % Line numbers start with line 1
        numberstyle=\tiny\color{Blue}, % Line numbers are blue and small
        stepnumber=5 % Line numbers go in steps of 5
}

% Creates a new command to include a perl script, the first parameter is the filename of the script (without .pl), the second parameter is the caption
\newcommand{\perlscript}[2]{
\begin{itemize}
\item[]\lstinputlisting[caption=#2,label=#1]{#1.pl}
\end{itemize}
}

%----------------------------------------------------------------------------------------
%	THEOREM STYLES
%----------------------------------------------------------------------------------------

\newtheoremstyle{colorhead}  % Name
{5pt}                        % Space above
{5pt}                        % Space below
{\itshape}                   % Body font
{}                           % Indent amount
{\color{Navy}\bfseries}      % Theorem head font
{.}                          % Punctuation after theorem head
{.5em}                       % Space after theorem head
{}                           % Theorem head spec (can be left empty, meaning "normal")

\theoremstyle{colorhead}
\newtheorem{zadatak}{Zadatak}
\newtheorem{primer}[zadatak]{Primer}
\newtheorem{kod}{Kod}

\newenvironment{resenje}{%
\renewcommand{\proofname}{\color{Red}Re\v senje}\renewcommand{\qedsymbol}{\color{Navy}$\square$}\proof}{\endproof}

%----------------------------------------------------------------------------------------
%	HEADERS AND FOOTERS
%----------------------------------------------------------------------------------------

% Headers - all currently empty
\lhead{}
\chead{}
\rhead{}

% Footers
\lfoot{}
\cfoot{}
\rfoot{\footnotesize Strana \thepage\ of \pageref{LastPage}} % "Page 1 of 2"

\renewcommand{\headrulewidth}{0.0pt} % No header rule
\renewcommand{\footrulewidth}{0.4pt} % Thin footer rule



%----------------------------------------------------------------------------------------
%	NASLOVNA STRANA
%----------------------------------------------------------------------------------------

\usepackage{titling}
\usepackage{xcolor}
\usepackage{sectsty}
\newcommand{\HorRule}{\color{DarkGoldenrod} \rule{\linewidth}{0.7pt}}
\pretitle{\vspace{-70pt} \begin{flushleft} \HorRule \fontsize{14}{14} \usefont{OT1}{phv}{m}{n} \color{DarkRed} \selectfont}
\title{Semestralni rad iz Diskretne matematike}
\posttitle{\par\end{flushleft}}

\preauthor{\begin{flushleft}\Large \usefont{OT1}{phv}{m}{sl} \color{DarkGoldenrod}}
\author{Tjuringova mašina \vskip 1.25em}

\postauthor{\Large \usefont{OT1}{phv}{m}{sl} \color{DarkRed}
Nikola Ilić, 2018/0063 \\
Matija Dodović, 2018/0072
\par\end{flushleft}



\vspace{-20pt}
\noindent
\HorRule}

%----------------------------------------------------------------------------------------

\begin{document}


\maketitle
 
\sectionfont{\color{black}}
\section{Uvod}

Tjuringova mašina je uređena sedmorka (S, $b$, $Q$, $q_0$, $q_+$, $q_-$, $f$), gde je $S$ azbuka, $b$ blanko znak (prazno polje), $Q$ skup stanja sa izdvojenim $q_0$ - početnim, i $q_+$ i $q_-$ - završnim stanjima sa pozitivnim, odnosno negativnim odgovorom. $f$ je program mašine i on je seldećeg oblika:     
\begin{center} \textbf{\textc{$f$($q_i$, $a_j$) = ($q_i'$, $a_j'$, $r$)}} \end{center} 
gde su $q_i$ i $q_i'$ trenutno i naredno stanje, a $a_j$ i $a_j'$ trenutni i novi znak. $r$ je iz skupa \{$+1, -1$\} i određuje da li se glava mašine pomera u levo ili u desno.\\
Program za realizaciju Tjuringove mašine učitava početnu traku i program, što je opisano na {\it Slici 2}. Traka može sadržati elemente bilo koje azbuke, uz ograničenje da mora početi i završiti se po jednim znakom $b$. Glava Tjuringove mašine je pozicionirana na poziciju $2$, uz indeksiranje pozicija od $1$, i ukoliko taj element bude prazan ($b$) program javlja grešku (što je ilustrovano na {\it Slici 4}). Pri pokretanju programa ispisuje se polazna traka. \\
Funkcija se zadaje u standardnom formatu, uz obavezno postojanje stanja $q_0$. Ne mogu postojati blanko znaci u funkciji. Ukoliko je izvršavanje programa korektno, ispisuju se traka i sledeći poziv funkcije. Ukoliko je odogovor pozitivan ispisuje se finalna traka u suprotnom se prijavljuje greška, {\it Slika 3}.

\section{Postupak korišćenja}
Na {\it Slici 1} je prikazan folder sa sadržajem potrebnim za pokretanje programa za izvršavanje Tjuringove mašine.
\begin{center}\includegraphics[width=15cm, height = 6.1cm]{folder.png}\\\caption{{\it Slika 1}: Prikaz foldera za pokretanje Tjuringove mašine.\end{center}\\
Egzekutabilni fajl $tjuringovaMasina.exe$ se pokreće na standardan način. U istom folderu kao i taj fajl je potrebno da se nalaze dva tekstualna fajla: $program.txt$ i $traka.txt$, gde su funkcija i traka, respektivno. Fajlovi su oblika opisanog u uvodu. Egzekutabilni fajl izvršava program Tjuringove mašine čitajući i obrađujući tekstualne datoteke i svoje izlaze ispisuju na konzoli ({\it Slike 2} i $3$). 
U folderu {\it test\_primeri} nalaze se zasebni folderi sa tri test primera izvršavanja programa Tjuringove mašine. U sva tri foldera su fajlovi sa funkcijom i trakom, koje treba kopirati u direktnorijum sa egzekutabilnim fajlom, radi njihovog pokretanja. Ukoliko se program izvrši sa pozitivnim odgovorom ispisuje se nova traka, a ukoliko je odgovor negativan izlazi poruka: "Greška na traci!".

\section{Prikaz rada koda}

\begin{center}\includegraphics[width=16.4cm, height = 8.2cm]{Osnovna.png}\\\caption{{\it Slika 2}: Prikaz {\it main} funkcije koda i primer po jedne datoteke {\it program.txt} i {\it traka.txt}. Pri pokretanju koda, linije koda 194-200 su zadužene za deklaracije i definicije, potrebne u kasnijem radu. Na liniji 202 se nalazi poziv procedure za učitavanje trake iz datoteke, koja se u naredne dve linije ispisuje (pozivom procedure {\it ispisi\_traku}). To je početna traka koja služi radi poređenja ispravnosti rada programa, tj izvršavanja funkcija. Na linijama 205 i 207 se nalaze pozivi procedura za učitavanje programa Tjuringove mašine (iz odgovarajuće datoteke) i njegove relaizacije, respektivno. Ispis međukoraka i finalne trake se nalazi u samoj proceduri za izvršavanje programa Tjuringove mašine.\end{center}\\

\begin{center}\includegraphics[width=16.4cm, height = 8.2cm]{izvrsavanjeKorektno.png}\\\caption{{\it Slika 3}: Primer korektnog izvršavanja programa koji ispituje da li je binaran broj palindrom. Prilikom učitavanja trake, koja počinje i završava se sa {\it b}, glava mašine pokazuje na broj (ne {\it b}). Izvršava se funkcija zadata u datoteci ({\it program.txt}), i u svakom koraku se ispisuje naredni poziv funkcije i trenutna traka. Pošto se poruka "Greška na traci!" nije ispisala, tj. u stanje {\it $q_-$} se nije ušlo, program se  korektno završio (stanje {\it $q_+$}) i ispisuje se promenjena traka, odakle se zaključuje da zadani broj jeste palindrom.  \end{center}\\ 

\begin{center}\includegraphics[width=16.4cm, height = 8.2cm]{izvrsavanjeNekorektno.png}\\\caption{{\it Slika 4}: Primer nekorektnog izvršavanja programa koji ispituje da li je binaran broj palindrom. U ovom primeru traka je zadata tako da glava pokazuje na element {\it b}. Po propozicijama, glava treba da pokazuje na početak broja, odnosno traka treba da bude zadata samo sa po jednim graničnim {\it b} znakom. Pokušava se sa identifikacijom nultog stanja i naredno stanje, usled glave koja pokazuje na {\it b}, jeste {\it $q_{-}$}. Ispisuje se traka do tog trenutka i poruka "Greška na traci!", uz prekidanje daljeg rada programa.\end{center}\\ 


\begin{kod}
Kod:
\lstinputlisting[language=C++]{tjuringovaMasina.cpp}
\end{kod}

\end{document} 